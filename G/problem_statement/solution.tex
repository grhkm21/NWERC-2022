\begin{frame}
    \frametitle{\problemtitle}
    \begin{block}{Problem}
        Determine the number $n$ of train carriages of a circular train using at most $3n + 500$ steps. \\
        In each step, you can either:
        \begin{itemize}
            \item move one carriage to the left,
            \item move one carriage to the right, or
            \item toggle the light switch in the current carriage.
        \end{itemize}
    \end{block}
    \pause
    \begin{block}{Solution}
        \begin{itemize}
            \item<+-> Naive solution: for some $x$, walk $x$ steps to the right
                turning everything off, then flip one light switch, and walk $x$ steps back to see if the light changed somewhere.
            \item<+-> If it did, then you know the length. If not, then try again with a larger $x$.
            \item<+-> This does not work: for small $x$, there is a lot of repetition so you need too many queries if $n$ is large. For large $x$, you use too many queries if $n$ is small.
        \end{itemize}
    \end{block}
\end{frame}

\begin{frame}
    \frametitle{\problemtitle}
    \begin{block}{Problem}
        Determine the number $n$ of train carriages of a circular train using at most $3n + 500$ steps.
    \end{block}
    \begin{block}{Solution}
        \begin{itemize}
            \item<+-> Alternative solution: use randomization.
            \item<+-> Choose a random sequence of bits of sufficient size (e.g. $25$).
            \item<+-> Set the initial $25$ bits to the chosen sequence.
            \item<+-> Walk to the right and keep track of the last read $25$ bits.
            \item<+-> If the last read bits correspond to the chosen sequence, we assume we made a full round.
            \item<+-> Determine the length of the round using the number of steps made.
        \end{itemize}
    \end{block}
\end{frame}


\begin{frame}
    \frametitle{\problemtitle}

    \begin{block}{Problem}
        Determine the number $n$ of train carriages of a circular train using at most $3n + 500$ steps.
    \end{block}
    \begin{block}{Pitfalls}
        \begin{itemize}
            \item<+-> The chosen bit sequence is not sufficiently long: De
                Bruijn sequences cover all $15$-bit patterns..
            \item<+-> The chosen bit sequence is not ``sufficiently random":
            \begin{itemize}
              \item \textsc{0000...}, \textsc{010101...},
              \item the default output of \texttt{rand()},
              \item the binary representation of special numbers: $\pi$, $e$, $\pi/2$, $\phi$.
            \end{itemize}
            \item<+-> Carefully handle the case where $n$ is smaller than the length of the chosen sequence!
        \end{itemize}
    \end{block}
    \solvestats

\end{frame}

